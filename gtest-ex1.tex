\subsection{Example1: vectint}

\begin{frame}[t]{A vector of integers}
\begin{itemize}
  \item Let's start with a simple class.
  \item A vector of integers with minimal functionality:
    \begin{itemize}
      \item A vector of integers with a size (\cppid{size()}) 
            and maximum capacity (\cppid{capacity()}).
      \item Support for copy and move.
      \item Access \cppkey{operator[]}.
      \item Capacity modification (\cppid{reserve()}). 
      \item Size modification (\cppid{resize()}). 
    \end{itemize}
\end{itemize}
\end{frame}

\begin{frame}[t]{Interface}
\begin{block}{include/vectint.h}
\lstinputlisting[lastline=15]{examples/vector1/include/vectint.h}
\end{block}
\end{frame}

\begin{frame}[t]{Interface}
\begin{block}{include/vectint.h}
\lstinputlisting[firstline=16,lastline=26]{examples/vector1/include/vectint.h}
\end{block}
\end{frame}

\begin{frame}[t]{Interface}
\begin{block}{include/vectint.h}
\lstinputlisting[firstline=27]{examples/vector1/include/vectint.h}
\end{block}
\end{frame}

\begin{frame}[t]{Implementation}
\begin{block}{src/vectint.cpp}
\lstinputlisting[lastline=12]{examples/vector1/src/vectint.cpp}
\end{block}
\end{frame}

\begin{frame}[t]{Implementation}
\begin{block}{src/vectint.cpp}
\lstinputlisting[firstline=13,lastline=27]{examples/vector1/src/vectint.cpp}
\end{block}
\end{frame}

\begin{frame}[t]{Implementation}
\begin{block}{src/vectint.cpp}
\lstinputlisting[firstline=29,lastline=43]{examples/vector1/src/vectint.cpp}
\end{block}
\end{frame}

\begin{frame}[t]{Implementation}
\begin{block}{src/vectint.cpp}
\lstinputlisting[firstline=44]{examples/vector1/src/vectint.cpp}
\end{block}
\end{frame}

\begin{frame}[t]{Program using vectint}
\begin{columns}[T]

\column{.5\textwidth}
\begin{block}{samples/main1.cpp}
\lstinputlisting[lastline=13]{examples/vector1/samples/main1.cpp}
\end{block}

\column{.5\textwidth}
\begin{block}{samples/main1.cpp}
\lstinputlisting[firstline=14]{examples/vector1/samples/main1.cpp}
\end{block}

\end{columns}
\end{frame}



