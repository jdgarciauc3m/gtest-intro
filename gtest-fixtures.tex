\subsection{Test Fixtures}

\begin{frame}[t]{What is a Fixture?}
\begin{itemize}
  \item A test case with the same object configuration for all tests.
    \begin{itemize}
      \item Allows to avoid writing repetitive code to set up tests.
    \end{itemize}

  \item How?
    \begin{itemize}
      \item Define a class with the name of the case derived from
            \cppid{::testing::Test}.
      \item Make data members for objects that must be visible from tests.
      \item Define constructor and/or destructor.
        \begin{itemize}
          \item Alternatively \cppid{SetUp()} and/or \cppid{TearDown()}.
        \end{itemize}
      \item Use macro \cppid{TEST\_F} instead of \cppid{TEST}.
    \end{itemize}
\end{itemize}
\end{frame}

\begin{frame}[t]{My first Fixture}
\begin{block}{vectint\_copy}
\lstinputlisting[lastline=12]{examples/vector4/utest/vectint_copy.cpp}
\end{block}
\end{frame}

\begin{frame}[t]{My first Fixture}
\begin{columns}

\column{.5\textwidth}
\begin{block}{vectint\_copy}
\lstinputlisting[firstline=14,lastline=23]{examples/vector4/utest/vectint_copy.cpp}
\end{block}

\column{.5\textwidth}
\begin{block}{vectint\_copy}
\lstinputlisting[firstline=14,lastline=23]{examples/vector4/utest/vectint_copy.cpp}
\end{block}

\end{columns}
\end{frame}
